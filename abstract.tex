% Options for packages loaded elsewhere
\PassOptionsToPackage{unicode}{hyperref}
\PassOptionsToPackage{hyphens}{url}
%
\documentclass[
]{article}
\usepackage{amsmath,amssymb}
\usepackage{iftex}
\ifPDFTeX
  \usepackage[T1]{fontenc}
  \usepackage[utf8]{inputenc}
  \usepackage{textcomp} % provide euro and other symbols
\else % if luatex or xetex
  \usepackage{unicode-math} % this also loads fontspec
  \defaultfontfeatures{Scale=MatchLowercase}
  \defaultfontfeatures[\rmfamily]{Ligatures=TeX,Scale=1}
\fi
\usepackage{lmodern}
\ifPDFTeX\else
  % xetex/luatex font selection
\fi
% Use upquote if available, for straight quotes in verbatim environments
\IfFileExists{upquote.sty}{\usepackage{upquote}}{}
\IfFileExists{microtype.sty}{% use microtype if available
  \usepackage[]{microtype}
  \UseMicrotypeSet[protrusion]{basicmath} % disable protrusion for tt fonts
}{}
\makeatletter
\@ifundefined{KOMAClassName}{% if non-KOMA class
  \IfFileExists{parskip.sty}{%
    \usepackage{parskip}
  }{% else
    \setlength{\parindent}{0pt}
    \setlength{\parskip}{6pt plus 2pt minus 1pt}}
}{% if KOMA class
  \KOMAoptions{parskip=half}}
\makeatother
\usepackage{xcolor}
\usepackage[margin=1in]{geometry}
\usepackage{graphicx}
\makeatletter
\def\maxwidth{\ifdim\Gin@nat@width>\linewidth\linewidth\else\Gin@nat@width\fi}
\def\maxheight{\ifdim\Gin@nat@height>\textheight\textheight\else\Gin@nat@height\fi}
\makeatother
% Scale images if necessary, so that they will not overflow the page
% margins by default, and it is still possible to overwrite the defaults
% using explicit options in \includegraphics[width, height, ...]{}
\setkeys{Gin}{width=\maxwidth,height=\maxheight,keepaspectratio}
% Set default figure placement to htbp
\makeatletter
\def\fps@figure{htbp}
\makeatother
\setlength{\emergencystretch}{3em} % prevent overfull lines
\providecommand{\tightlist}{%
  \setlength{\itemsep}{0pt}\setlength{\parskip}{0pt}}
\setcounter{secnumdepth}{-\maxdimen} % remove section numbering
\ifLuaTeX
  \usepackage{selnolig}  % disable illegal ligatures
\fi
\usepackage{bookmark}
\IfFileExists{xurl.sty}{\usepackage{xurl}}{} % add URL line breaks if available
\urlstyle{same}
\hypersetup{
  hidelinks,
  pdfcreator={LaTeX via pandoc}}

\author{}
\date{\vspace{-2.5em}}

\begin{document}

\section*{Apresentação}\label{apresentauxe7uxe3o}
\addcontentsline{toc}{section}{Apresentação}

\begin{center}

\begin{minipage}{.7\linewidth}
O Boletim de Conjuntura Econômica do Estado do Tocantins é uma das
atividades do Grupo \abbr{pet} de Ciências Econômicas da {[}abbr:@uft{]}
e tem como objetivo apresentar a evolução das principais variáveis
macroeconômicas do estado. Esta edição tem o formato com dados
trimestrais e mensais de 2023, estando a periodicidade das informações
limitada à divulgação de dados pelas fontes oficiais e organizações.
Este ano, mais uma vez contamos com a parceria do Conselho Regional de
Economia ({[}abbr:@coreconto{]}). As informações contidas são destinadas
a cidadãos, gestores públicos e empresários, sendo provenientes de
fontes oficiais de organizações públicas.

Os textos e as análises apresentados têm caráter informativo. Os
comentários não refletem obrigatoriamente os posicionamentos públicos do
{[}abbr:@coreconto{]} ou da {[}abbr:@uft{]}. As análises podem ou não
sofrer alterações, caso se confirmem, em função da revisão de dados
pelas fontes no que concerne ao período da análise, a mudanças na
conjuntura econômica e social decorrentes de atos governamentais e a
forças exógenas, como, por exemplo, o caso da pandemia da COVID-19. O
momento com a pandemia se tornou um desafio para as sociedades
brasileira e mundial.

Neste número, o Boletim traz dados sobre o Produto Interno Bruto
({[}abbr:@pib{]}), contas públicas, taxa de pobreza, coeficiente de
Gini, mercado de trabalho, comércio exterior e agricultura. O
{[}abbr:@pib{]} corresponde à soma de toda a riqueza de uma nação num
determinado período de tempo. Nesta edição, apresentamos o
{[}abbr:@pib{]} pelo lado da demanda e da oferta. Pelo lado da demanda,
ele é constituído pela soma do consumo das famílias, governo,
investimentos e exportações líquidas; pelo lado da oferta, ele é
constituído pela soma de tudo o que é produzido por todos os setores.

As contas públicas estaduais, compreendem as receitas e as despesas do
governo. As receitas podem ser provenientes de tributos, transferências,
contribuição e de outras fontes, e as despesas, de diferentes setores,
como saúde, educação, pessoal, indústria, entre outros. Inclui-se também
a capacidade de pagamento do Estado, sua situação fiscal, que compreende
endividamento, poupança corrente e liquidez. No campo social, temos a
taxa de pobreza e o Índice de Gini. O coeficiente de Gini é uma medida
utilizada para calcular a desigualdade na distribuição de renda. Varia
entre zero e um: zero significa completa igualdade de renda e um,
completa desigualdade. Por consequência, quanto mais próximo de um,
maior é a concentração de renda.

A variável Emprego corresponde ao número de pessoas ocupadas
formalmente. Apresenta o perfil do empregado (idade, gênero, etnia, grau
de instruções), o saldo de emprego do Tocantins e da Região Norte bem
como os setores de contratação e demissão, seguro desemprego e
rendimento médio. O tópico comércio exterior traz a análise descritiva
dos dados do saldo comercial em dólares de 2023. Apresenta os principais
produtos exportados e importados e os países com os quais o Tocantins
tem relação comercial. A agricultura apresenta informações sobre soja,
milho e arroz bem como informações sobre a pecuária, em especial, a
bovinocultura.

\hfill\break
Prof.~Dr.~Nilton Marques de Oliveira -- Tutor {[}abbr:@pet{]} Ciências
Econômicas

\end{minipage}

\end{center}

\thispagestyle{empty}
\twocolumn
\pagenumbering{arabic}

\end{document}
