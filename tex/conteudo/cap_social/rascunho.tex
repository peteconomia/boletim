\chapter{Social}
\par Para identificar os níveis de pobreza de uma população, é primordial a classificação de aspectos para um padrão de vida digno e satisfatório, como dieta balanceada, vestimentas adequadas, acesso a serviços de saúde e educação, ambiente sadio, etc.
\par No Brasil, a Constituição Federal do Brasil de 1988, garante no Art. 6º que todos têm direitos sociais, estabelecendo dimensões para o bem-estar da população, como a educação, a saúde, a alimentação, o trabalho, a moradia, o transporte, o lazer, a segurança, a previdência social, a proteção à maternidade e à infância e a assistência aos desamparados.
\par Neste sentido, cabe-nos a pergunta: Como andam os indicadores sociais tocantinenses levando em conta os últimos anos em que se teve um baixo crescimento do produto e consequentemente uma deterioração do mercado de trabalho? Assim, iniciamos apresentando no gráfico abaixo a evolução da taxa de pobreza do nosso estado entre 2012 e 2018, comparando com o desempenho da região Norte e com o Brasil.
\begin{figure}[h]
	\caption{Taxa de Pobreza - Linha de US\$5,50}
	\includegraphics[width=\linewidth]{fig/taxa_pobreza.pdf}
	\source{IBGE}
	\notes{Elaboração própria}
\end{figure}
\par Apesar do contexto apresentado na pergunta anterior, a taxa de pobreza do Tocantins apresentou uma queda de 38,67\% para 31,54\%, o que em números absolutos representou uma queda de 12,42\%. Uma queda expressiva, ainda mais se comparada à média dos estados da região Norte, havendo inclusive um aumento da diferença com o nosso estado ao longo dos últimos anos. Já se comparada à taxa brasileira, a taxa tocantinense ainda é maior, porém houve uma diminuição dessa diferença, uma vez que a taxa do nosso país não apresentou grandes oscilações nos anos analisados.

\begin{smbox}[label={labelbox},nameref={Taxas de Pobreza}]{Taxas de Pobreza}
	Uma das formas mais comuns de se mensurar pobreza é através de uma linha de pobreza absoluta, que define como pobres aqueles que vivem com uma renda inferior ao valor adotado pela linha. Neste sentido, o Banco Mundial sugere linhas que se adaptam melhor para as condições de vida de determinados países. Para um país de rendimento médio-alto, é sugerido uma linha de US\$5,50 PPC.  
\end{smbox}
\par Porém, se olharmos para uma faixa de renda menor ainda, a de extremamente pobres, os resultados não seguiram a mesma tendência, indicando um maior impacto do cenário apresentado para menores faixas de renda. Os resultados são apresentados no gráfico abaixo.
\begin{figure}[h]
	\caption{Taxa de Pobreza - Linha de US\$5,50}
	\includegraphics[width=\linewidth]{fig/taxa_expobreza.pdf}
	\source{IBGE}
	\notes{Elaboração própria}
\end{figure}
\par A taxa de extrema pobreza apresentou uma leve alta entre 2012 e 2018, saindo de 5,59\% para 6,59\%. Em termos absolutos de pessoas vivendo nessa condição, temos a mínima em 2014 onde a partir daí, ocorre uma alta de 34,04\%, um detalhe que em muitas vezes pode passar desapercebido se olharmos somente para a taxa. O mesmo comportamento pode ser observado no indicador para o Brasil e para região Norte.
\begin{figure}[h]
	\caption{Índice de Gini}
	\includegraphics[width=\linewidth]{fig/gini.pdf}
	\source{IBGE}
	\notes{Elaboração própria}
\end{figure}
\begin{smbox}[label={labelbox},nameref={Índice de Gini}]{Índice de Gini}
	É um índice que demonstra o grau de concentração de renda de um determinado grupo. Seus resultados variam entre 0 e 1. Quanto mais próximo de 0, mais igual é aquele grupo, e quanto mais próximo de 1, mas desigual é aquele grupo.
\end{smbox}
\par É possível perceber que houve uma leve alta do índice no nosso estado ao longo dos anos apresentados, saindo de 0,509 para 0,528, seguindo a tendência dos outros dados apresentados até então. Para o Brasil e para região Norte, idem. 
\par Os resultados apresentados nessa seção são produto, como já mencionado, do baixo crescimento econômico na década de 2010 e as suas consequências no mercado de trabalho, com aumento da taxa de desemprego, precarização dos trabalhos e aumento do trabalho informal. A crise fiscal enfrentada pela União e pelo estado do Tocantins de certa forma também contribuem para esse quadro, uma vez que gastos com programas sociais são muitas vezes cortados em contextos como este. Isso sem falar da baixa qualidade dos serviços públicos ofertados para a parte da população mais necessitada, o que de certa forma, perpetua o quadro apresentado aqui.
\par Superar as ainda altas taxas de pobreza e desigualdade devem fazer parte de uma agenda para o nosso estado do Tocantins, para que possamos ter uma economia mais forte, que cresça de forma mais sustentável e regular e para que o nosso povo viva cada vez melhor.  
