\chapter*{Sumário Executivo}
\addcontentsline{toc}{chapter}{Sumário Executivo}
\begin{center}
	\parbox{0.7\linewidth}{
		\par O Boletim de Conjuntura do Estado do Tocantins 2019 apresenta as variáveis: Produto Interno Bruto (PIB), Emprego, Orçamento Público, Agropecuária e Indicadores Sociais para o Estado do Tocantins e, em alguns casos, para a região Norte. 
		\par O Produto Interno Bruto corresponde à soma de toda a produção pela economia de um determinado lugar, dado um determinado período de tempo. Sua composição setorial segue a tradicional divisão em setores primário, secundário e terciário, aqui também chamados de agropecuária, indústria e comércio e serviços, respectivamente. A variável PIB foi considerada para o período de 2007 a 2016, com análises dos dados microrregionais do estado, de sua composição setorial e de sua evolução recente. A fonte dos dados relativos à variável Produto Interno Bruto é o Instituto Brasileiro de Geografia e Estatística/IBGE. O Produto Interno Bruto per capita corresponde à razão entre o Produto Interno Bruto e a população de um determinado território.  
		\par A variável Emprego corresponde ao número de pessoas ocupadas formalmente em 31 de dezembro do respectivo ano, sendo uma variável de estoque, foi considerada para o período de 2007 a 2017. Além da evolução e das taxas de crescimento, são apresentadas as participações dos Setores (Grandes Setores de Atividades pela Classificação do Instituto Brasileiro de Geografia e Estatística) e das Microrregiões (segundo classificação do Instituto Brasileiro de Geografia e Estatística) na composição do Emprego total do estado. Os dados de Emprego foram coletados junto ao Ministério do Trabalho e Emprego/MTE, a partir da Relação Anual de Informações Sociais/RAIS. 
		\par O Orçamento Público perfaz as receitas e despesas do governo do estado, em um dado período de tempo. As receitas podem advir de tributos, transferências, contribuição e outras. Já as despesas podem se realizar em diferentes setores, como saúde, educação, pessoal, indústria, entre outros. Os orçamentos públicos estaduais seguem o mesmo padrão do orçamento nacional, de modo que neste tópico serão discutidas algumas das principais receitas e despesas estaduais tocantinenses durante o período de 2009 a 2018, a partir dos dados do Finanças no Brasil/FINBRA. 
		\par Já o tópico Agropecuária apresenta as informações sobre a cultura da soja, milho, entre outros produtos agrícolas, bem como informações sobre a pecuária, em especial a bovinocultura. O relatório apresenta os dados de 2017. A base de dados foi obtida a partir da pesquisa da Produção Agrícola Municipal (PAM), do IBGE.
	}
\end{center}
\thispagestyle{empty}
